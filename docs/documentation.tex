\documentclass[]{article}

\usepackage{algorithm}
\usepackage{algpseudocode}
\usepackage{amsmath}
\usepackage{mathtools}
\usepackage{amsfonts}
%\usepackage{physics}
\usepackage{bm}

\usepackage[dvipsnames]{xcolor}
\usepackage{pgfplots} %import after xcolor!!!
\usepackage{geometry}
\geometry{
	a4paper,
	total={170mm,257mm},
	left=20mm,
	top=15mm,
	bottom=15mm
}

\usepackage[font=small,labelfont=bf]{caption}
\usepackage{subcaption}
\usepackage{graphicx}
\graphicspath{ {./Figures/} }


\title{Theoretical Background}
\author{Namu Kroupa}
\date{\today}

\begin{document}
	\maketitle
	
	\section{Matrix equation}
	To calculate the reflectivity, $R=|b_0|^2$, the matrix equation 
	\begin{equation}\label{eqn:matrix-equation}
		\mathbf{M}\mathbf{x}=\mathbf{c}
	\end{equation}
	must be solved for $\mathbf{x}$, where
	\begin{equation}
		\mathbf{x}=(b_0,a_1,b_1,a_2,b_2,\dots,a_M,b_M,a_{M+1})^T
	\end{equation}
	and
	\begin{equation}
		\mathbf{c}=(1,1,0,0,\dots,0)^T
	\end{equation}
	are $(2M+2)$-dimensional complex vectors.
	
	The matrix $\mathbf{M}\in\mathbb{C}^{(2M+2)\times(2M+2)}$ is 
	\begin{equation}
		\mathbf{M}=
		\begin{pmatrix}
			\begin{matrix}a\\b\end{matrix}&\bm{\beta}_0&\bm{0}&\mathbf{0}&\mathbf{0}&\cdots&\mathbf{0}&\begin{matrix}0\\0\end{matrix}\\
			\hline
			\begin{matrix}0\\0\end{matrix}&\bm{\alpha}_1&\bm{\beta}_1&\mathbf{0}&\mathbf{0}&\cdots&\mathbf{0}&\begin{matrix}0\\0\end{matrix}\\
			\hline 
			\begin{matrix}0\\0\end{matrix}&\mathbf{0}&\bm{\alpha}_2&\bm{\beta}_2&\mathbf{0}&\cdots&\mathbf{0}&\begin{matrix}0\\0\end{matrix}\\
			\hline
			\begin{matrix}0\\0\end{matrix}&\mathbf{0}&\mathbf{0}&\bm{\alpha}_3&\bm{\beta}_3&\cdots&\mathbf{0}&\begin{matrix}0\\0\end{matrix}\\
			\hline
			\vdots&\vdots&\vdots&\vdots&\vdots&\ddots&\bm{\beta}_{M-1}&\begin{matrix}0\\0\end{matrix}\\
			\hline
			\begin{matrix}0\\0\end{matrix}&\mathbf{0}&\mathbf{0}&\mathbf{0}&\mathbf{0}&\cdots&\bm{\alpha}_{M}&\begin{matrix}c\\d\end{matrix}
		\end{pmatrix},
	\end{equation}
	(the horizontal lines are for readability) where 
	\begin{equation}
		\mathbf{0}=\begin{pmatrix}0&0\\0&0\end{pmatrix}.
	\end{equation}

	To solve Equation \ref{eqn:matrix-equation} efficiently, its ``band-storage form'' $\mathbf{M}_\text{band}\in \mathbb{C}^{5\times(2M+2)}$ is used:
	\begin{equation}
		\mathbf{M}_\text{band}=
		\begin{pmatrix}
			\begin{matrix}0\\0\\a\\b\\0\end{matrix}&\bm{c}_0&\bm{c}_1&\cdots&\bm{c}_{M-1}&\begin{matrix}0\\c\\d\\0\\0\end{matrix}
		\end{pmatrix}
	\end{equation}
	where
	\begin{equation}
		\bm{c}_i=
		\begin{pmatrix}
			0&(\bm{\beta}_i)_{01}\\
			(\bm{\beta}_i)_{00}&(\bm{\beta}_i)_{11}\\
			(\bm{\beta}_i)_{10}&(\bm{\alpha}_{i+1})_{01}\\
			(\bm{\alpha}_{i+1})_{00}&(\bm{\alpha}_{i+1})_{11}\\
			(\bm{\alpha}_{i+1})_{10}&0
		\end{pmatrix}\quad\quad\text{for}\quad i=0,\dots,M-1
	\end{equation}

	For s-polarisation,
	\begin{align}
		a&=-1\\
		b&=1\\
		c&=1\\
		d&=(k_x)_\text{substrate}\\
		\bm{\alpha}_j&=
		\begin{pmatrix}
			-e^{i\phi_{j-1}}&-1\\
			-e^{i\phi_{j-1}}(k_x)_{j-1}&(k_x)_{j-1}
		\end{pmatrix}\quad\quad\text{for}\quad j=1,\dots,M\\
		\bm{\beta}_j&=
		\begin{dcases}
			\begin{pmatrix}
				1&e^{i\phi_0}\\
				\frac{(k_x)_0}{(k_x)_\text{outer}}&-\frac{(k_x)_0}{(k_x)_\text{outer}}e^{i\phi_0}
			\end{pmatrix}&\text{for}\quad j=0\\
			\begin{pmatrix}
				1&e^{i\phi_j}\\
				(k_x)_j&-(k_x)_je^{i\phi_j}
			\end{pmatrix}&\text{for}\quad j=1,\dots,M-1
		\end{dcases}.
	\end{align}

	For p-polarisation,
	\begin{align}
		a&=-1\\
		b&=1\\
		c&=\frac{(k_x)_\text{substrate}}{n_\text{substrate}}\\
		d&=n_{\text{substrate}}\\
		\bm{\alpha}_j&=
		\begin{pmatrix}
			-e^{i\phi_{j-1}}\frac{(k_x)_{j-1}}{n_{j-1}}&-\frac{(k_x)_{j-1}}{n_{j-1}}\\
			-e^{i\phi_{j-1}}n_{j-1}&n_{j-1}
		\end{pmatrix}\quad\quad\text{for}\quad j=1,\dots,M\\
		\bm{\beta}_j&=
		\begin{dcases}
			\begin{pmatrix}
				\frac{(k_x)_0}{n_0}\frac{n_\text{outer}}{(k_x)_\text{outer}}&e^{i\phi_0}\frac{(k_x)_0}{n_0}\frac{n_\text{outer}}{(k_x)_\text{outer}}\\
				\frac{n_0}{n_\text{outer}}&-e^{i\phi_0}\frac{n_0}{n_\text{outer}}
			\end{pmatrix}&\text{for}\quad j=0\\
			\begin{pmatrix}
				\frac{(k_x)_j}{n_j}&e^{i\phi_j}\frac{(k_x)_j}{n_j}\\
				n_j&-e^{i\phi_j}n_j
			\end{pmatrix}&\text{for}\quad j=1,\dots,M-1
		\end{dcases}.
	\end{align}
	
	\section{Fabry-Perot interferometer}
	As a test case, the Fabry-Perot interferometer is considered, which consists of a single layer of refractive index $n_\text{layer}$ and thickness $d$ in a medium of refractive index $n_\text{outer}$ and whose reflectivity at a wavelength $\lambda_\text{vac}$ (in vacuum) and incident angle $\theta_\text{outer}$ is given by
	\begin{equation}
		R=\left|\frac{r(1-e^{2i\phi_0})}{1-r^2 e^{2i\phi_0}}\right|^2
	\end{equation}
	where 
	\begin{align}
		\phi_0&=\frac{2\pi}{\lambda_\text{vac}}n_\text{layer}d\cos\theta_\text{layer}\\
		r&=
		\begin{dcases}
		\frac{n_\text{outer}\cos\theta_\text{outer}-n_\text{layer}\cos\theta_\text{layer}}{n_\text{outer}\cos\theta_\text{outer}+n_\text{layer}\cos\theta_\text{layer}}&\text{for s-polarisation},\\
		\frac{n_\text{layer}\cos\theta_\text{outer}-n_\text{outer}\cos\theta_\text{layer}}{n_\text{layer}\cos\theta_\text{outer}+n_\text{outer}\cos\theta_\text{layer}}&\text{for p-polarisation}
		\end{dcases}.
	\end{align}
	The angle $\theta_\text{layer}$ is the angle of the light ray inside the layer. Its cosine is given by
	\begin{equation}
		\cos\theta_\text{layer}=\sqrt{1-\left(\frac{n_\text{outer}\sin\theta_\text{outer}}{n_\text{layer}}\right)^2}.
	\end{equation}
	In the case of total internal refraction, $\cos\theta_\text{layer}$ becomes imaginary and the above equations for $R$, $\phi_0$ and $r$ continue to hold.
	
%	\bibliographystyle{plain}
%	\bibliography{refs}
	
\end{document}
