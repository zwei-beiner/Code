\documentclass[]{article}

\usepackage{algorithm}
\usepackage{algpseudocode}
\usepackage{amsmath}
\usepackage{mathtools}
\usepackage{amsfonts}
%\usepackage{physics}
\usepackage{bm}

\usepackage[dvipsnames]{xcolor}
\usepackage{pgfplots} %import after xcolor!!!
\usepackage{geometry}
\geometry{
	a4paper,
	total={170mm,257mm},
	left=20mm,
	top=15mm,
	bottom=15mm
}

\usepackage[font=small,labelfont=bf]{caption}
\usepackage{subcaption}
\usepackage{graphicx}
\graphicspath{ {./Figures/} }

% For multiple footnotes separated by a comma
\usepackage[multiple]{footmisc}


\title{Documentation}
\author{Namu Kroupa}
\date{\today}

\begin{document}
	\maketitle
	
	\section{Calculation of the reflected amplitude}
	\subsection{Matrix equation}
	\begin{figure}[H]
		\centering
		\includegraphics[width=0.8\textwidth]{multilayer_stack.png}
		\caption{}
		\label{fig:multilayer-stack}
	\end{figure}
	
	{\small
	(Notation in this Section: The multilayer stack consists of $M$ layers, numbered from $j=0$ to $j=M-1$. The outer medium from which the light is incident is denoted by $\text{``outer''}$ and the substrate into which the light exits is denoted by $\text{``substrate''}$. See Figure \ref{fig:multilayer-stack}.)}
	
	To calculate the reflectivity, $R=|b_\text{outer}|^2$, the matrix equation 
	\begin{equation}\label{eqn:matrix-equation}
		\mathbf{M}\mathbf{x}=\mathbf{c}
	\end{equation}
	must be solved for $\mathbf{x}$, where
	\begin{equation}
		\mathbf{x}=(b_\text{outer},a_0,b_0,a_1,b_1,\dots,a_{M-1},b_{M-1},a_\text{substrate})^T
	\end{equation}
	and
	\begin{equation}
		\mathbf{c}=(1,1,0,0,\dots,0)^T
	\end{equation}
	are $(2M+2)$-dimensional complex vectors.
	
	The matrix $\mathbf{M}\in\mathbb{C}^{(2M+2)\times(2M+2)}$ is\footnote{The horizontal lines are for readability.}
	\begin{equation}
		\mathbf{M}=
		\begin{pmatrix}
			\begin{matrix}a\\b\end{matrix}&\bm{\beta}_0&\bm{0}&\mathbf{0}&\mathbf{0}&\cdots&\mathbf{0}&\begin{matrix}0\\0\end{matrix}\\
			\hline
			\begin{matrix}0\\0\end{matrix}&\bm{\alpha}_0&\bm{\beta}_1&\mathbf{0}&\mathbf{0}&\cdots&\mathbf{0}&\begin{matrix}0\\0\end{matrix}\\
			\hline 
			\begin{matrix}0\\0\end{matrix}&\mathbf{0}&\bm{\alpha}_1&\bm{\beta}_2&\mathbf{0}&\cdots&\mathbf{0}&\begin{matrix}0\\0\end{matrix}\\
			\hline
			\begin{matrix}0\\0\end{matrix}&\mathbf{0}&\mathbf{0}&\bm{\alpha}_2&\bm{\beta}_3&\cdots&\mathbf{0}&\begin{matrix}0\\0\end{matrix}\\
			\hline
			\vdots&\vdots&\vdots&\vdots&\vdots&\ddots&\bm{\beta}_{M-1}&\begin{matrix}0\\0\end{matrix}\\
			\hline
			\begin{matrix}0\\0\end{matrix}&\mathbf{0}&\mathbf{0}&\mathbf{0}&\mathbf{0}&\cdots&\bm{\alpha}_{M-1}&\begin{matrix}c\\d\end{matrix}
		\end{pmatrix},
	\end{equation}
	where 
	\begin{equation}
		\mathbf{0}=\begin{pmatrix}0&0\\0&0\end{pmatrix}.
	\end{equation}

	To solve Equation \ref{eqn:matrix-equation} efficiently, its ``band-storage form'' $\mathbf{M}_\text{band}\in \mathbb{C}^{5\times(2M+2)}$ is used:
	\begin{equation}
		\mathbf{M}_\text{band}=
		\begin{pmatrix}
			\begin{matrix}0\\0\\a\\b\\0\end{matrix}&\bm{c}_0&\bm{c}_1&\cdots&\bm{c}_{M-1}&\begin{matrix}0\\c\\d\\0\\0\end{matrix}
		\end{pmatrix}
	\end{equation}
	where
	\begin{equation}
		\bm{c}_i=
		\begin{pmatrix}
			0&(\bm{\beta}_i)_{01}\\
			(\bm{\beta}_i)_{00}&(\bm{\beta}_i)_{11}\\
			(\bm{\beta}_i)_{10}&(\bm{\alpha}_{i})_{01}\\
			(\bm{\alpha}_{i})_{00}&(\bm{\alpha}_{i})_{11}\\
			(\bm{\alpha}_{i})_{10}&0
		\end{pmatrix}\quad\quad\text{for}\quad i=0,\dots,M-1
	\end{equation}

	For s-polarisation,
	\begin{align}
		a&=-1\\
		b&=1\\
		c&=1\\
		d&=n_\text{substrate}\cos\theta_\text{substrate}\\
		\bm{\alpha}_j&=
		\begin{pmatrix}
			-e^{i\phi_{j}}&-1\\
			-e^{i\phi_{j}}n_{j}\cos\theta_{j}&n_{j}\cos\theta_{j}
		\end{pmatrix}\quad\quad\text{for}\quad j=0,\dots,M-1\\
		\bm{\beta}_j&=
		\begin{dcases}
			\begin{pmatrix}
				1&e^{i\phi_0}\\
				\frac{n_0\cos\theta_0}{n_\text{outer}\cos\theta_\text{outer}}&-\frac{n_0\cos\theta_0}{n_\text{outer}\cos\theta_\text{outer}}e^{i\phi_0}
			\end{pmatrix}&\text{for}\quad j=0\\
			\begin{pmatrix}
				1&e^{i\phi_j}\\
				n_j\cos\theta_j&-n_j\cos\theta_je^{i\phi_j}
			\end{pmatrix}&\text{for}\quad j=1,\dots,M-1
		\end{dcases}.
	\end{align}

	For p-polarisation,
	\begin{align}
		a&=-1\\
		b&=1\\
		c&=\cos\theta_\text{substrate}\\
		d&=n_{\text{substrate}}\\
		\bm{\alpha}_j&=
		\begin{pmatrix}
			-e^{i\phi_{j}}\cos\theta_{j}&-\cos\theta_{j}\\
			-e^{i\phi_{j}}n_{j}&n_{j}
		\end{pmatrix}\quad\quad\text{for}\quad j=0,\dots,M-1\\
		\bm{\beta}_j&=
		\begin{dcases}
			\begin{pmatrix}
				\frac{\cos\theta_0}{\cos\theta_\text{outer}}&e^{i\phi_0}\frac{\cos\theta_0}{\cos\theta_\text{outer}}\\
				\frac{n_0}{n_\text{outer}}&-e^{i\phi_0}\frac{n_0}{n_\text{outer}}
			\end{pmatrix}&\text{for}\quad j=0\\
			\begin{pmatrix}
				\cos\theta_j&e^{i\phi_j}\cos\theta_j\\
				n_j&-e^{i\phi_j}n_j
			\end{pmatrix}&\text{for}\quad j=1,\dots,M-1
		\end{dcases}.
	\end{align}
	where 
	\begin{align}
		\phi_j&=k_\text{outer}d_j\frac{n_j}{n_\text{outer}}\cos\theta_{j}\\
		\cos\theta_j&=\sqrt{1-\left(\frac{n_\text{outer}\sin\theta_\text{outer}}{n_j}\right)^2}.
	\end{align}

	\subsection{Fabry-Perot interferometer}
As a test case, the Fabry-Perot interferometer is considered, which consists of a single layer of refractive index $n_\text{layer}$ and thickness $d$ in a medium of refractive index $n_\text{outer}$ and whose complex reflected amplitude, $b_{0,\text{FP}}$, at a wavelength $\lambda_\text{vac}$ (in vacuum) and incident angle $\theta_\text{outer}$ is given by\footnote{The sign in the complex exponential can be positive or negative depending on the convention. But it must be positive here to be consistent with Equation \ref{eqn:matrix-equation}.}\footnote{This expression allows $r$ and $\phi_0$ to become complex, as occurs in total internal reflection, as opposed to the commonly found expression for the reflectivity, which assumes that $r$ is real.}
\begin{equation}
	b_{0,\text{FP}}=\frac{r(1-e^{2i\phi_0})}{1-r^2 e^{2i\phi_0}}
\end{equation}
where\footnote{The sign of $r_p$ depends on convention, specifically if the electric field amplitudes were initially chosen to be parallel or antiparallel in the derivation of $r_{s/p}$. For example, Hecht and Wikipedia give the negative of the expression used in this text. For consistency with Equation \ref{eqn:matrix-equation}, $r_p$ must be chosen as presented here.}
\begin{align}
	\phi_0&=\frac{2\pi}{\lambda_\text{vac}}n_\text{layer}d\cos\theta_\text{layer}\\
	r&=
	\begin{dcases}
		\frac{n_\text{outer}\cos\theta_\text{outer}-n_\text{layer}\cos\theta_\text{layer}}{n_\text{outer}\cos\theta_\text{outer}+n_\text{layer}\cos\theta_\text{layer}}&\text{for s-polarisation},\\
		\frac{n_\text{outer}\cos\theta_\text{layer}-n_\text{layer}\cos\theta_\text{outer}}{n_\text{layer}\cos\theta_\text{outer}+n_\text{outer}\cos\theta_\text{layer}}&\text{for p-polarisation}
	\end{dcases}.
\end{align}
The angle $\theta_\text{layer}$ is the angle of the light ray inside the layer. Its cosine is given by
\begin{equation}
	\cos\theta_\text{layer}=\sqrt{1-\left(\frac{n_\text{outer}\sin\theta_\text{outer}}{n_\text{layer}}\right)^2}.
\end{equation}
In the case of total internal refraction, $\cos\theta_\text{layer}$ becomes imaginary and the above equations for $R$, $\phi_0$ and $r$ continue to hold.

	\subsection{Transfer-matrix method}
	\begin{figure}[H]
		\centering
		\includegraphics[width=0.5\textwidth]{transfer_matrix.png}
		\caption{}
		\label{fig:transfer-matrices}
	\end{figure}

	{\small (Notation in this Section: The outer medium has index $0$, the first and last layer of the multilayer stack have index $1$ and $M$, respectively, and the substrate has index $M+1$.)}
	
	An alternative approach to calculate $b_0$ is the transfer-matrix method. It can be shown that, at interface $i$ at position $x_i$ between layers $i$ and $j$ ($i<j$), the electric field amplitudes of the forward (F) and backward (B) travelling waves on the left ($x_i^-$) and right ($x_i^+$), where $x_i^-<x_i^+$, (Figure \ref{fig:transfer-matrices}) are related by
	\begin{equation}
		\begin{pmatrix}
			E_F(x_j^-)\\
			E_B(x_j^-)
		\end{pmatrix}=
		\mathbf{T}_{ij}
		\begin{pmatrix}
			E_F(x_j^+)\\
			E_B(x_j^+)
		\end{pmatrix}
	\end{equation} 
	where 
	\begin{equation}
		\mathbf{T}_{ij}=
		\frac{1}{t_{ij}}
		\begin{pmatrix}
			1&r_{ij}\\
			r_{ij}&1
		\end{pmatrix}.
	\end{equation}
	Furthermore, within layer $i$, the electric field amplitudes on the left and right are related by complex exponentials such that
	\begin{equation}
		\begin{pmatrix}
			E_F(x_{i-1}^+)\\
			E_B(x_{i-1}^+)
		\end{pmatrix}=
		\mathbf{T}_i
		\begin{pmatrix}
			E_F(x_i^-)\\
			E_B(x_i^-)
		\end{pmatrix}
	\end{equation}
	where 
	\begin{equation}
		\mathbf{T}_i=
		\begin{pmatrix}
			e^{-i\Phi_i}&0\\
			0&e^{i\Phi_i}
		\end{pmatrix}
	\end{equation}
	and 
	\begin{equation}
		\Phi_i=\frac{2\pi}{\lambda_\text{vac}}n_id_i\cos\theta_i.
	\end{equation}
	The Fresnel equations are 
	\begin{align}
		r_{ij}&=
		\begin{dcases}
			\frac{n_i \cos\theta_i - n_j \cos\theta_j}{n_i \cos\theta_i + n_j \cos\theta_j}&\text{for s-polarisation}\\
			\frac{n_i \cos\theta_j-n_j \cos\theta_i}{n_j \cos\theta_i + n_i \cos\theta_j}&\text{for p-polarisation}
		\end{dcases}\\
		t_{ij}&=
		\begin{dcases}
			\frac{2 n_i \cos\theta_i}{n_i \cos\theta_i + n_j \cos\theta_j}&\text{for s-polarisation}\\
			\frac{2 n_i \cos\theta_i}{n_j \cos\theta_i + n_i \cos\theta_j}&\text{for p-polarisation}
		\end{dcases}.
	\end{align}
	Hence, the equation relating the incident, $(1, b_0)^T$, to the transmitted field amplitudes, $(a_{M+1},0)^T$, are
	\begin{equation}
		\begin{pmatrix}
			1\\b_0
		\end{pmatrix}=
		\mathbf{T}
		\begin{pmatrix}
			a_{M+1}\\0
		\end{pmatrix}
	\end{equation}
	where the transfer matrix, $\mathbf{T}$, is defined as\footnote{Reminder: ``Layer'' $0$ is the outer medium, layers $1$ and $M$ are the first and last layers on the multilayer stack, respectively, and ``layer'' $M+1$ is the substrate.}
	\begin{equation}
		\mathbf{T}=\mathbf{T}_{01}\mathbf{T}_1\mathbf{T}_{12}\mathbf{T}_2\dots\mathbf{T}_{M}\mathbf{T}_{M(M+1)}.
	\end{equation}
	Thus, the reflected amplitude, $b_0$, is related to the transfer matrix components by
	\begin{equation}
		b_0=\frac{(\mathbf{T})_{10}}{(\mathbf{T})_{00}}.
	\end{equation}

%\section{The merit function}
%
%A multilayer coating is specified by the refractive index $n_i(\lambda)$ and thickness $d_i$ of each layer. These parameters are collectively denoted by
%\begin{equation}
%	\bm{p}=(n_1, n_2, \dots, n_M, d_1, d_2, \dots, d_M).
%\end{equation}
%

	
%	\bibliographystyle{plain}
%	\bibliography{refs}
	
\end{document}
